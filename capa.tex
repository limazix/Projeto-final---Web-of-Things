\begin{titlepage}
\clearpage
\bigskip


\bigskip



\bigskip

\centerline{\textbf{\large \emph{Web of Things:} Uma Prova de Conceito com Aplicação em Acessibilidade}}


\bigskip


\bigskip

\centerline{\textbf{Bruno Lima Cardoso}}



\bigskip


\bigskip

Projeto Final de Curso submetido ao Departamento de Ci\^encia da Computa\c{c}\~ao do Instituto de Matem\'atica da Universidade Federal do Rio de Janeiro como parte dos requisitos necess\'arios para obten\c{c}\~ao do grau de Bacharel em Ciêcia da Computação.


\bigskip

Apresentado por: Bruno Lima Cardoso


\bigskip


%\hfill
%\_\_\_\_\_\_\_\_\_\_\_\_\_\_\_\_\_\_\_\_\_\_\_\_\_\_\_\_\_
\hfill
\line(1,0){250}

\hfill
Bruno Lima Cardoso

\bigskip

\hfil
Aprovado por:

\bigskip

\bigskip

\hfill
\line(1,0){250}

\hfill
Prof. Gabriel P. Silva, D. Sc.

\hfill
(Presidente)

\bigskip

\bigskip

\bigskip

\hfill
\line(1,0){250}

\hfill
Prof. José Antonio Borges, D. Sc.

\bigskip

\bigskip

\bigskip

\hfill
\line(1,0){250} 

\hfill
Prof. Fernando Gil Vianna Resende Junior, Ph. D.

\bigskip

\bigskip

\bigskip

\bigskip

\begin{center}

RIO DE JANEIRO, RJ - BRASIL

AGOSTO DE 2012

\end{center}

\bigskip

\clearpage
\bigskip

\begin{center}
\centerline{\textbf{\large RESUMO}}


\bigskip

\emph{Web of Things:} Uma Prova de Conceito com Aplicação em Acessibilidade


\bigskip

Bruno Lima Cardoso


\end{center}
\bigskip


\bigskip


Orientador: Gabriel P. Silva 


\bigskip


\bigskip

Este trabalho apresenta um sistema que propicia maior independência funcional aos portadores de lesão motora severa, para que dispensem o auxílio de terceiros para interagir com o ambiente ao seu redor. Para tal, foi desenvolvido um protótipo de um sistema de automação residencial cujo objetivo é virtualizar o acesso a aparelhos eletrônicos como, televisão, sistema de som e similares,  possibilitando que o deficiente os acesse diretamente. Utilizando Web of Things como arquitetura base, será mostrado que este conceito é possível e extremamente viável. Desta forma, foi desenvolvido um protótipo possuidor de três camadas: rede de sensores e atuadores (virtualizador dos aparelhos), o Smart Gateway (que padroniza o acesso aos aparelhos) e a aplicação (um controlador pertencente ao usuário). Nesta última camada em especial, será apresentada uma implementação de uma tecnologia amplamente utilizada para acessibilidade virtual, e que está em franco crescimento: o reconhecimento de voz. Ao final, serão discutidas algumas peculiaridades do protótipo apresentado, tais como, curva de aprendizagem, grau de complexidade de desenvolvimento e abordagens tomadas que poderiam ser melhoradas. Entretanto, provar o conteúdo teórico apresentado aqui está além do escopo deste projeto, mas espera-se que o trabalho apresentado sirva de base para futuros desenvolvimentos e gere novas discussões sobre a arquitetura, o sistema e suas diversas aplicações.

\bigskip

\clearpage
\bigskip

\begin{center}
\centerline{\textbf{\large ABSTRACT}}


\bigskip

\emph{Web of Things:} Uma Prova de Conceito com Aplicação em Acessibilidade


\bigskip

Bruno Lima Cardoso


\end{center}
\bigskip


\bigskip

Supervisor: Gabriel P. Silva


\bigskip

This paper presents a system that provides greater functional independence to people with severe motor injury, to waive the assistance of others to interact with the environment around them. For this porpouse, was developed a prototype of a home automation system whose goal is to virtualize access to electronic devices such as TV, sound system and others, allowing directly access to them. Using Web of Things as basic architecture will be shown that this concept is very possible and feasible. This prototype has three layers: wireless actuator and sensor network (virtualizer apparatus), the Smart Gateway (which standardizes access to equipment) and application (a user controller). At this last layer in particular, will be presented an implementation of a technology widely used for virtual accessibility: the voice recognition. Then, will be discussed some peculiarities of the prototype presented, such as the learning curve, development complexity level and approaches taken that could be improved. However, to prove the statements given here is beyond the scope of this paper, but it could be the basis of further and hopefully intensive discussion of system architecture and its various applications.

\bigskip


\bigskip


\bigskip
\end{titlepage}
