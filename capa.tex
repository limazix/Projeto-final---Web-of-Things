\begin{titlepage}
\clearpage
\bigskip


\bigskip



\bigskip

\centerline{\textbf{\large \emph{Web of Things:} Uma Prova de Conceito com Aplicação em Acessibilidade}}


\bigskip


\bigskip

\centerline{\textbf{Bruno Lima Cardoso}}



\bigskip


\bigskip

Projeto Final de Curso submetido ao Departamento de Ci\^encia da Computa\c{c}\~ao do Instituto de Matem\'atica da Universidade Federal do Rio de Janeiro como parte dos requisitos necess\'arios para obten\c{c}\~ao do grau de Bacharel em Ciêcia da Computação.


\bigskip

Apresentado por: Bruno Lima Cardoso


\bigskip


%\hfill
%\_\_\_\_\_\_\_\_\_\_\_\_\_\_\_\_\_\_\_\_\_\_\_\_\_\_\_\_\_
\hfill
\line(1,0){250}

\hfill
Bruno Lima Cardoso

\bigskip

\hfil
Aprovado por:

\bigskip

\bigskip

\hfill
\line(1,0){250}

\hfill
Prof. Gabriel P. Silva, Ph. D.

\hfill
(Presidente)

\bigskip

\bigskip

\bigskip

\hfill
\line(1,0){250}

\hfill
Prof. José Antonio Borges, Ph. D.

\bigskip

\bigskip

\bigskip

\hfill
\line(1,0){250} 

\hfill
Prof. Fernando Gil Vianna Resende Junior, Ph. D.

\bigskip

\bigskip

\bigskip

\bigskip

\begin{center}

RIO DE JANEIRO, RJ - BRASIL

JULHO DE 2012

\end{center}

\bigskip

\clearpage
\bigskip

\begin{center}
\centerline{\textbf{\large RESUMO}}


\bigskip

\emph{Web of Things:} Uma Prova de Conceito com Aplicação em Acessibilidade


\bigskip

Bruno Lima Cardoso


\end{center}
\bigskip


\bigskip


Orientador: Gabriel P. Silva 


\bigskip


\bigskip

Este trabalho apresenta uma possível estrutura que propicie maior independência funcional à portadores de lesão severa que necessitam de auxílio de terceiros para interagir com o ambiente ao seu redor. Para tal, será desenvilvido um protótipo de sistema de automação residencial cujo intúito é virtualize o acesso aos aparelhos eletrônicos como, teleisão, sistema de son e similares, à fim de possibilitar que o deficiente os acesse diretamente. Utilizando \emph{Web of Things} como arquitetura base, será mostrado que este conceito é possível e extremamente viável. Com este objetivo o protótipo possui três camadas: rede de sensores e atuadores - virtualizador dos aparelhos, o \emph{Smart Gateway} - padroniza o acesso aos aparelhos - e a aplicação - controlador pertencente ao usuário. Neste última camada em especial, será demonstrado como funciona uma tecnologia amplamente utilizada para acessibilidade virtual, e que está em franco crescimento: o reconhecimento de voz.

Ao final, serão discutidas algumas peculiaridades do protótipo apresentado, tais como, curva de aprendizagem, grau de complexidade de desenvolvimento e abordagens tomadas que podem ser melhorada.

Provar o conteúdo teórico apresentado aqui está além do escopo deste projeto, mas espera-se que sirva de base para futuros trabalhos e intensas discussões sobre a arquitetura, o sistema e as diversas aplicações.

\bigskip

\clearpage
\bigskip

\begin{center}
\centerline{\textbf{\large ABSTRACT}}


\bigskip

\emph{Web of Things:} Uma Prova de Conceito com Aplicação em Acessibilidade


\bigskip

Bruno Lima Cardoso


\end{center}
\bigskip


\bigskip

Supervisor: Gabriel P. Silva


\bigskip

\ \ Text that presents an abstract of the work giving a general description.


\bigskip


\bigskip


\bigskip
\end{titlepage}
