\begin{titlepage}
\clearpage
\bigskip


\bigskip



\bigskip

\centerline{\textbf{\large \emph{Web of Things:} Uma Prova de Conceito com Aplicação em Acessibilidade}}


\bigskip


\bigskip

\centerline{\textbf{Bruno Lima Cardoso}}



\bigskip


\bigskip

Projeto Final de Curso submetido ao Departamento de Ci\^encia da Computa\c{c}\~ao do Instituto de Matem\'atica da Universidade Federal do Rio de Janeiro como parte dos requisitos necess\'arios para obten\c{c}\~ao do grau de Bacharel em Ciêcia da Computação.


\bigskip

Apresentado por: Bruno Lima Cardoso


\bigskip


%\hfill
%\_\_\_\_\_\_\_\_\_\_\_\_\_\_\_\_\_\_\_\_\_\_\_\_\_\_\_\_\_
\hfill
\line(1,0){250}

\hfill
Bruno Lima Cardoso

\bigskip

\hfil
Aprovado por:

\bigskip

\bigskip

\hfill
\line(1,0){250}

\hfill
Prof. Gabriel P. Silva, D. Sc.

\hfill
(Presidente)

\bigskip

\bigskip

\bigskip

\hfill
\line(1,0){250}

\hfill
Prof. José Antonio Borges, D. Sc.

\bigskip

\bigskip

\bigskip

\hfill
\line(1,0){250} 

\hfill
Prof. Fernando Gil Vianna Resende Junior, Ph. D.

\bigskip

\bigskip

\bigskip

\bigskip

\begin{center}

RIO DE JANEIRO, RJ - BRASIL

AGOSTO DE 2012

\end{center}

\bigskip

\clearpage
\bigskip

\begin{center}
\centerline{\textbf{\large RESUMO}}


\bigskip

\emph{Web of Things:} Uma Prova de Conceito com Aplicação em Acessibilidade


\bigskip

Bruno Lima Cardoso


\end{center}
\bigskip


\bigskip


Orientador: Gabriel P. Silva 


\bigskip


\bigskip

Este trabalho apresenta um sistema que propicia maior independência funcional aos portadores de lesão motora severa, para que dispensem o auxílio de terceiros para interagir com o ambiente ao seu redor. Para tal, foi desenvolvido um protótipo de um sistema de automação residencial cujo objetivo é virtualizar o acesso a aparelhos eletrônicos como, televisão, sistema de som e similares,  possibilitando que o deficiente os acesse diretamente. Utilizando Web of Things como arquitetura base, será mostrado que este conceito é possível e extremamente viável. O protótipo desenvolvido possui três camadas: rede de sensores e atuadores (virtualizador dos aparelhos), o Smart Gateway (que padroniza o acesso aos aparelhos) e a aplicação (um controlador pertencente ao usuário). Nesta última camada em especial, será apresentada uma implementação de uma tecnologia amplamente utilizada para acessibilidade virtual, e que está em franco crescimento: o reconhecimento de voz. Ao final, serão discutidas algumas peculiaridades do protótipo apresentado, tais como, curva de aprendizagem, grau de complexidade de desenvolvimento e abordagens tomadas que poderiam ser melhoradas. Provar o conteúdo teórico apresentado aqui está além do escopo deste projeto, mas espera-se que o trabalho apresentado sirva de base para futuros desenvolvimentos e gere novas discussões sobre a arquitetura, o sistema e suas diversas aplicações.

\bigskip

\clearpage
\bigskip

\begin{center}
\centerline{\textbf{\large ABSTRACT}}


\bigskip

\emph{Web of Things:} Uma Prova de Conceito com Aplicação em Acessibilidade


\bigskip

Bruno Lima Cardoso


\end{center}
\bigskip


\bigskip

Supervisor: Gabriel P. Silva


\bigskip

\ \ Text that presents an abstract of the work giving a general description.


\bigskip


\bigskip


\bigskip
\end{titlepage}
